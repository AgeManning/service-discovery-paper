%!TEX root = ../main.tex
%=========================================================

\section{Introduction}
\para{Motivation} Ethereum~\cite{buterin2013ethereum}  is one of the largest permissionless,  open-source  blockchains supporting Turing-complete scripting via smart contracts. Ethereum runs a Peer-to-Peer (P2P) network based on Kademlia Distributed Hash Table (DHT) ~\cite{maymounkov2002kademlia} to exchange pending transactions and mined blocks. 

%Following its decentralized design, Ethereum relies on a Peer-to-Peer (P2P) communication network, where individual nodes, running an Ethereum client software (\eg Go Ethereum~\cite{go-ethereum}), send and receive messages containing transactions and blocks to achieve distributed consensus, without relying on any central node or a trusted third party. The core system and the associated protocols used by the peers to participate in the Ethereum P2P network is included as part of a suite called DEVp2p. The suite includes individual sub-systems and protocols for services such as node discovery or establishing secure or point-to-point communication between peers. 

Apart from supporting blockchain data exchange, the Ethereum P2P network is also used by other applications. This includes multiple Ethereum testnets (Ropsten, Rinkeby),  alternative cryptocurrencies(Pirl, Musicoin) and other decentralized platforms such as iExec or Swarm.
In 2018, the Ethereum P2P network nodes operated a total of 4,076 different applications\cite{kim2018measuring}. Most of those applications require a dedicated P2P network to disseminate application-specific messages (\eg event notifications) to the interested peers but rely on the larger, Ethereum P2P network for bootstrapping.

Using an existing, well-established P2P network to bootstrap smaller, application specific networks has multiple advantages. Firstly, it encourages re-using reliable code to perform common tasks such as securing connections or finding peers. Secondly, P2P network security increases with the number of its honest participants. Being part of a larger swarm makes smaller applications more resistant to various attacks\hl{some refs from attack of smaller blockchains}. Finally, application developers can focus uniquely on implementing application-specific services boosting the growth of the entire, decentralised ecosystem. 

To realise this vision, nodes must \textit{(i)} join a larger, host P2P network, \textit{(ii)} discover their application-specific peers, \textit{(iii)} form an application-specific network. However, peer discovery in a fully decentralised setup is a challenging task. Unlike existing discovery protocols that work in a local setting (MDNS/Bonjour\hl{[?]}) or rely on trusted operators \hl{[?]}, decentralised node discovery must work at a global scale and in a fully decentralised setting. Due to the high heterogeneity of applications and (potentially constraint) peers, such a protocol must be efficient and introduce low communication and computational overhead. Finally, an open system, must protect against malicious actors trying to disturb network services, exhaust resources of honest peers or hijack application-specific networks. 

\para{State of the art} In the current Ethereum service discovery system, \ie discovery version 4 (Discv4), nodes perform a \textit{brute-force} discovery process. A node willing to discover application-specific peers randomly traverses the DHT and initiates connections with all the encountered nodes. A permanent connection is established only if a peer supports the same application and terminated otherwise. Such an approach provides strong security guarantees but results in a large communication overhead and long discovery times, particularly for unpopular applications with a small number of peers. While multiple, alternative discovery systems have been proposed \hl{[?]}, they rely on unrealistic assumptions \hl{[?]}, introduce high overhead or fail to operate in a presence of an adversary \hl{[?]}. 

\para{Discv5} In this work, we propose a new \textbf{service discovery sub-system} which enables secure,  efficient and robust  discovery of application-specific peers in the Ethereum network,  that will be part of the new DEVp2p  discovery system version 5 (\textit{\sysname}).
Different from Discv4, \sysname allows nodes (\ie advertisers) to associate themselves with a set of \emph{topics} (\eg application IDs), and advertise the association (\ie an ad) to the network. The information is collectively stored by network participants (\ie registrars) without relying on a single trusted party at any point. Any node can then query the network for a topic to obtain a list of application-specific peers and directly connect to them without disturbing other members of the network. 

We build \sysname on top of the existing Ethereum DHT to avoid major changes to the system and take advantage of the already existing routing infrastructure. The protocol propagates application-specific advertisements to multiple nodes chosen in an unpredictable way to protect against targeted attacks. Our design encourages diversity of advertisements stored at each registrar making the system resistant to network dynamics. At the same time, \sysname provides efficient peer discovery operations terminating within a bounded number of steps for all the topics regardless of their popularity. The protocol implements a robust admission mechanism to limit the resource usage on all the nodes, ensure fairness across topics, and prevent a vast range of malicious behaviours even in the presence of a powerful attacker.

\para{Contributions} We make the following contributions:
\begin{itemize}
    \item In \Cref{sec:placement}, we design a DHT-based data placement system that distributes service advertisement in the network. The protocol combines pseudo-random data placement for security with deterministic operations for efficiency. We present a lookup operation that finds a subset of advertisement placed in the system within a bounded amount of time. The procedure ensure the diversity of data sources and is resistant against manipulation by malicious registrars. 
    \item In \Cref{sec:registration} we design a lightweight admission protocol allowing advertisers to place ads after waiting for a specified amount of time. \sysname guarantees that advertisers cannot place more advertisement by deviating from the protocol and does not create any intermediary state at nodes holding the advertisements. 
    \item In \Cref{sec:waitingTime}, we design a function that calculates a waiting time after which advertisement can be placed on nodes holding advertisements. The function limits the amount of resources used by each node, promotes ads diversity stored within nodes, and protects against a vast range of malicious behaviours. 
\end{itemize}

In \Cref{sec:eval}, we evaluate \sysname under different system parameters and against state-of-the-art decentralised discovery systems. Our system achieves better load-balancing and requires more resource consumption by the attackers to successfully launch attacks such as eclipsing attacks. We also provide performance evaluation results from both simulations and from real system deployment. \sysname is scheduled for implementation in the next version of the Ethereum P2P network. 

%\michal{From the paragraph below we should form a requirements parts}
% An open and decentralized discovery system introduces new security challenges that need to be addressed. Malicious nodes may decide to deviate from the protocol in order to increase their own visibility or disrupt system operations. A particularly important security challenge is the \textit{Sybil attack} which can amplify the attacking power of malicious nodes, typically with very little resource consumption. The Sybils can launch various attacks including Denial-of-Service (DoS) by ignoring ad lookups, spam ads with bogus topic advertisement, and launch more sophisticated attacks to isolate nodes from their peers in their target sub-networks (\ie eclipse attack). 
%\michal{The above needs completion}
%\ramin{Is the above information about attacks needed in the introduction?}
%\michal{Probably not. We must just say there are problems with security}



