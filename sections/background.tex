%!TEX root = ../main.tex
%=========================================================

\section{Background}
\label{sec:background}

\michal{Need a good introduction on Kademlia here. Distances/buckets etc. It's required to explain the ticket table, routing etc.}
\etienne{How different is the Ethereum Kademlia from the original published version? I can assume there have been changes in so many years. Is there a reference document for the Ethereum's Kademlia I can read?}
\sergi{From what i remember main difference are num of buckets and use of log distance.  I don't remember any document but Felix will help.}
\sergi{We should also add how kademlia is used for discovery (when lookups are performed, lookup buffer use,  pool of connections, random lookups, etc}

\subsection{Ethereum DHT}
Ethereum DHT is based on Kademlia - a UDP-based, peer-to-peer distributed hash table (DHT) that is used to locate data stord in a decentralized way~\cite{maymounkov2002kademlia}. Each node is uniquely identified by its randomly generated node ID. A data item stored in the DHT is found by its key, which is simply a hash of the data itself. Node IDs and data keys share the same representation; in the following we use the term ID for both.

Kademlia leverages this by storing data at nodes whose node ID is “close” to the data’s key. Closeness is defined as the bitwise XOR between two IDs, taken as an integer value, i.e.,$d(x, y) = \textit{log}_2 \lfloor x \oplus y \rfloor$.


. A node stores its known neighbors in so-called k-buckets which partition the known network based on the local node’s ID. Every k-bucket (or simply bucket) stores up to $k$ neighbors. Intuitively, node IDs are treated as the leaves of a binary tree and each bucket stores a distinct branch of the tree. Bucket $i$ stores nodes whose distance is in $[2i, 2i+1)$, which effectively corresponds to the length of the common prefix between two node IDs. Items in the DHT and nodes on the network are located by \emph{lookups}. A lookup successively queries nodes that are closest to the desired target ID (key or node ID).

Node IDs in Ethereum DHT serve as public identifiers for each node in the Ethereum network. A node ID in Ethereum is a marshaled 512-bit ECDSA public key. However, distance computations only operate on Keccak256- hashes [11] of node IDs, effectively yielding node IDs with 256-bit length. When referring to node IDs in the following we implicitly mean hashed ECDSA public keys. It is easy to generate many different identities but computationally infeasible to generate a specific one.