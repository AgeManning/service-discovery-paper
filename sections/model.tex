%!TEX root = ../main.tex
%=========================================================

\section{System and threat models}
\label{sec:model}

%\michal{Say we build on top of the DHT. What do we need from the DHT?}
%\michal{Those are speerate layers.}
In this section, we present the network and threat models under which the \sysname discovery layer (\Cref{sec:overview}) operates. 

\subsection{System Model}
We assume a network of nodes. At startup time, each node generates a public/secret key pair, which it uses to secure point-to-point communication with its peers. Nodes are identified by their \emph{node ID}, which is simply the hash of their public key. We assume that multiple nodes may share the same IP address (due to NAT or being hosted by the same physical machine). However, two nodes cannot share the same ID.

Our system, \sysname, is built on the existing Ethereum DHT\footnote{At the time of writing, the Ethereum DHT consist of 3,500-5,000 nodes.}, \ie nodes use the Ethereum DHT to bootstrap and initially discover nodes to fill their tables. Specifically, we designed and implemented our system as an extension of Ethereum's \emph{Node Discovery Protocol v5 (discv5)}. However, the operation of service discovery is not Ethereum-specific and could also be implemented using a different DHT.

\sysname indexes participants according to \emph{topics}. A topic is an identifier for a
service or application provided by a node. Topics are 256-bit integers, like node IDs.
A node providing a certain service (topic), is said to \emph{register} an advertisement for itself when it registers the ad on a registrar to make itself discoverable under that topic. Depending on the needs of the application, a node can advertise multiple topics or no topics at all. We assume that the popularity of topics in the system may vary significantly and follows a power law distribution~\cite{kim2018measuring}. Anyone can participate in registering and searching for (one or more) topics and uses the same ID and IP for all its topics.

\felix{Can we remove this role list?}

Nodes fulfill the following roles in the network:

\begin{itemize}
    \item \textbf{Registrar} - accepts advertisements and responds to topic queries. When asked for a specific topic, a registrar responds with nodes that registered ads for the topic.  All DHT nodes act as Registrars. A Registrar accepts and stores advertisements for any topic.
    \item \textbf{Advertiser} - registers for a specific topic and wants to be discovered by its peers. The advertisers make themselves discoverable by placing advertisements. Nodes are advertisers for every topic/service it provides.
    \item \textbf{Searcher} - attempts to discover nodes for a specific topic.
\end{itemize}

\subsection{Threat Model}
Our system is designed to operate in an adversarial environment, \ie. the Internet. We assume the presence of malicious actors in the DHT. These actors do not strictly follow the protocol when communicating with others and attempt to influence the discovery results of honest nodes by steering them toward the attacker-controlled nodes.

Security of the service discovery mechanism against attacks is fundamentally dependent on the security guarantees provided by the underlying DHT implementation. This is because the DHT's basic protocol is used to initialize and maintain the data structures used by the service discovery layer. We assume that no honest node is fully eclipsed by the malicious ones, \ie each honest DHT node has at least one honest peer. 

Malicious actors can spawn multiple virtual nodes within one physical machine and thus control many Sybil node IDs. Maintaining nodes in the DHT requires infrastructure resources (public IP addresses) and we assume the resources under the control of an attacker are limited. Specifically, we assume that it is easier for an attacker to generate similar IP addresses (\ie within a single subnet) than it is to control many diverse IP addresses (with different prefixes). We use the number of IP addresses and IDs under the control of an attacker as parameters for our evaluation in Section \ref{sec:eval}.

Through an extensive literature review\cite{chen2020survey, henningsen2019eclipsing}, we collect a list malicious behaviours that an attacker can use to disrupt a DHT-based service discovery protocol:
\begin{itemize}
    \item \textbf{Malicious DHT Peer} - DHT routing relies on asking peers for other nodes, closer to a specific target (\eg MSG\_FIND, MSG\_GET). When responding to those messages, an attacker returns uniquely other malicious nodes trying to hijack the DHT traversal process of honest nodes.
    \item \textbf{Malicious Registrar} - when queried for a specific topic, it returns a maximum amount of malicious nodes \footnote{Alternatively, a malicious registrar could simply refuse to respond. However, such behaviour is less effective than returning malicious peers}. 
    \item \textbf{Spamming Advertisers} - an attacker bombards honest registrars with malicious advertisers. The attack can be performed for a single or multiple topics. The attack may cause an honest registrar to refuse honest advertisement due to the lack of resources (both storage and CPU power), and return malicious advertisements of the spammer when queried by honest searchers. 
    \item \textbf{Generic Spammers} - an attacker bombards an honest registrar with topic queries or random traffic hoping to exhaust bandwidth or CPU power. However, we consider such attacks less of a threat as: \textit{(i)} they require a symmetric amount of resources used by the attacker to establish and maintain a secure connection, \textit{(ii)} can be easily countered by sending rate limits (\eg based on the source IP address or Proof-of-Work puzzles). 
    \item \michal{We also have some \sysname-specific attacks (\eg trying to return high waiting times to slow down/stop an honest registration process. However, it's too early to introduce them here. Maybe we can just analyse it in a discussion section after introducing the protocol?}
\end{itemize}

Adversaries can strategically target specific nodes or regions in the DHT ID space (\ie by generating Sybil Node IDs within the desired region of the DHT ID space) with those attacks. We assume presence of attackers that can combine some or all of the behaviours above and coordinate their actions to maximize their effect. 


%\subsection{Security}
%Adversaries can generate Sybils at little cost and launch various attacks to disrupt discovery of and to eclipse peers of target applications.


%In a structured P2P network,  The success of such regional attacks is closely related to the load-balancing performance of \sysname: if the registrations for any topic favor specific regions in the DHT (\ie creating hotspots), then strategically targeting the hotspot regions can significantly amplify the strength of such attacks as we demonstrate in Section \hl{X}.


% As creating a node requires maintaining periodic, encrypted communication with its peers, the number of active IDs an attacker can posses at a time is limited.

%
%
%\sergi{we assume this or is it a requirement for the design? Maybe last sentence shouldn't go in this section \michal{I'd say both. We assume this and it's a requirement for our design. IMO, this is something we take for granted and don't propose anything to enhance it. Note, it's about eclipsing DHT nodes not about the the eclipse attack within a topic that we investigate later.}}

%The attacks on \sysname involve adversaries (and their Sybils) abusing their advertiser and registrar roles:

%\dk{It might be worth mentioning that even with perfect protection against Sybil attacks, a powerful attacker controlling a certain percentage of the network (Tor attacker model) can mount eclipse attacks.}
%\mk{That's a good point, we should probably just target making Sybil attacks costly/difficult rather then impossible (which actually cannot be achieved}

%XXX Onur: we might want to mention that we don't worry about malicious (e.g., spamming) searchers and explain why.
